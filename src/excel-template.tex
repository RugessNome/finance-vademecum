
\chapter{Démontrer la rentabilité d'un investissement}

\chapterprecishere{``N'oubliez pas le template !''\par\raggedleft--- Hugues Gissler}

Dans cette partie, on se propose de montrer l'utilisation d'Excel 
pour étudier la rentabilité d'un investissement.
Cette partie est à lire dans la continuité du chapitre 2 où l'on introduit 
la notion de taux d'actualisation.

Pour étudier la rentabilité d'un investissement sur $n$ années, 
on va considérer:
\begin{itemize}
  \item ce que l'investissement nous coûte au départ;
  \item ce qu'il nous rapporte chaque année (produits);
  \item ce qu'il nous coûte chaque année (charges).
\end{itemize}

De cela, on en déduit un flux de trésorerie pour chaque année. 
Dès que l'on considére des années futures, il est nécessaires d'actualiser 
les flux de trésoreries. Il faut pour cela choisir une valeur pour le taux 
d'actualisation (par exemple 10\%).

On peut donc construire un tableau Excel ayant la forme suivante.

\begin{center}
\begingroup
\scriptsize
\begin{tabular}{lccccc}
  Année          & 0   & 1     & 2     & $\cdots$ & $n$ \\
  \hline
  Produits       & 0   & $x_1$ & $x_2$ & $\cdots$ & $x_n$ \\
  Charges        & 0   & $y_1$ & $y_2$ & $\cdots$ & $y_n$ \\
  Investissement & $I$ & 0     & 0     & $\cdots$ & 0 \\
  \hline
  Flux tréso.\/  & $I$ & $f_1$ & $f_2$ & $\cdots$ & $f_5$ \\
  Flux actualisé & $I$ & $g_1$ & $g_2$ & $\cdots$ & $g_n$ \\
\end{tabular}
\endgroup
\end{center}

On a les formules suivantes:
\begin{itemize}
  \item $f_i = x_i + y_i$;
  \item $g_i = f_i \times (1 + \alpha^i)^{-1}$.
\end{itemize}

Il est possible de calculer le taux de rentabilité interne en 
sélectionnant toutes les valeurs de la ligne des flux de trésoreries 
(non actualisés) et en utilisant la fonction \textsc{TRI} d'Excel.

Si l'on somme les flux de trésoreries actualisés (fonction \textsc{SOMME}), 
on obtient la VAN (valeur actuelle nette). 
Cette valeur donne une première indication de la rentabilité puisque si elle 
est positive cela signifie que l'investissement est rentable.
Une autre quantité très intéressante se calcul en divisant la VAN 
sur l'investissement initial, il s'agit du ROI.
\[
\text{ROI} = \frac{\text{VAN}}{-I}
\]

Pour aller plus loin, on peut décomposer les lignes correspondant aux produits et 
aux charges en sous-lignes et introduire une incertitude sur ces quantités.
Excel nous permet de construire un tableau permettant de faire des simulations 
pour différentes valeurs de 2 variables.
On dit que l'on fait une analyse de sensibilité.

\begin{center}
\begingroup
\scriptsize
\begin{tabular}{l|ccccc}
  $v$      & $a_1$   & $a_2$  & $\cdots$ & $a_n$ \\
  \hline
  $b_1$    &         &        &          & \\
  $b_2$    &         &        &          &  \\
  $\vdots$ &         &        &          &  \\
  $b_n$    &         &        &          &
\end{tabular}
\endgroup
\end{center}

Avec:
\begin{itemize}
  \item dans la case $v$, la formule permettant de calculer la quantité d'intérêt (par exemple la VAN);
  \item $a_i$, les différentes valeurs de la première variable;
  \item $b_i$, les différentes valeurs de la seconde variable.
\end{itemize}

La procédure est alors la suivante:
\begin{enumerate}
  \item sélectionner l'ensemble du tableau;
  \item aller dans l'onglet \textit{Données}, puis cliquer sur \textit{Analyse de scénarios}
        puis \textit{Table de données};
  \item entrer dans le champs \textit{Cellule d'entrée en ligne} le nom de la case contenant 
        la valeur de la première variable;
  \item faire de même pour la deuxième variable et cliquer sur \textsc{ok}.
\end{enumerate}
Le tableau de remplit alors avec les valeurs calculées. 
Pour un résultat plus visuel, il est possible de demander à Excel de colorier les 
cases en allant dans l'onglet \textit{Accueil} et en cliquant sur \textit{Mise en forme conditionnelle}.