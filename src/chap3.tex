
\chapter{Une activité indésirable ?}

\chapterprecishere{``La qualité !''\par\raggedleft--- Un membre de l'option Finance}

Je viens d'être embauché dans une petite entreprise de 
fabrication en tant qu'ingénieur qualité. Cette société 
familiale utilise ses nombreuses machines pour fabriquer 
du mobilier métalique et d'autres pièces de A à Z. \\
\hspace*{\parindent}En plus des machines de découpage et de pliage et des postes  
de soudage et de peinture, la société possède une grenailleuse :
il s'agit d'une machine capable d'envoyer sur les pièces du 
sable à haute vitesse pour éliminer les dépôts de rouille ou 
enlever une couche de peinture. \\
\hspace*{\parindent}Cette activité est la bête noire des employés car elle est 
extrêmement pénible. On commence par remplir un réservoir de 
grenaille (que l'on ramasse à même le sol), puis on 
enfile un scaphandre de projection et on projette la grenaille 
sur les pièces à l'aide d'une lance ! \\
\hspace*{\parindent}La machine étant viellisante, son efficacité a diminuée. 
De plus, le scaphandre de protection n'est plus tout à fait 
hermétique et la machine fait beaucoup de bruit ce qui rend 
le travail d'autant plus difficile à supporter. \\
\hspace*{\parindent}Pire encore, le comptable de la société est perturbé dans son 
travail d'enregistrement des pièces comptables par le bruit de 
la grenailleuse. Les finances de l'entreprise sont toujours 
justes, la société ayant fait le choix de la qualité sur la 
quantité, les clients se font rares...\footnote{Les clients 
sont néanmoins en général très satisfait du travail de la 
société.} \\ 
\hspace*{\parindent}Le comptable décide donc de faire des calculs pour voir si on 
ne pourrait pas faire des économies. A sa grande surprise, il 
découvre que le grenaillage est l'une des activités les moins 
rentables, à la limite de faire perdre de l'argent.
\begin{itemize}
 \item "Il faut qu'on arrête ça", se dit-il alors.
\end{itemize}
Ni une ni deux, le comptable sort alors de son bureau,
grimpe les escaliers, cours vers le bureau du patron. 
Il rentre dans le bureau tout essoufflé, il est peu habitué 
à faire du sport. 
\begin{itemize}
  \item "Chef, il faut qu'on parle du grenaillage"
\end{itemize}
Le chef, bien au courant de ce qu'il se passe dans son atelier, 
de lui répondre: "Oui oui je sais, il faut réinvestir dans la 
machine!".
Le financier tombe alors des nues puisque le patron vient de 
proposer exactement l'inverse de ce qu'il voulait suggérer.


Avant d'aller plus loin, il faut comprendre comment le 
comptable est arrivé à la conclusion que l'activité n'était pas 
rentable: nous allons pour cela faire un calcul de marge.

On peut regrouper les charges en 3 grandes catégories: 
\begin{itemize}
  \item les charges variables, liées au volume d'activité;
  \item les charges fixes spécifiques, ne dépendent pas du volume 
        d'activité mais disparaîtraient si on cessait l'activité;
  \item les charges communes, ne sont pas liées à une activité particulière.
\end{itemize}

Pour l'activité de grenaillage, l'entreprise achète chaque année 10 kg de grenailles 
à 60\euro\/ le kilo pour compenser les pertes. La fiche technique de la grenailleuse 
affiche un fonctionnement nominal sur du 400 V - 32 A. 
La prestation est facturée à l'heure, le tarif est fixé à 50\euro\/ l'heure. \\
Le comptable commence donc ses calculs. 
La machine consomme 12,8 kW; en prenant un coût de l'électricité à 14 centimes du kWh, 
on obtient 1,80\euro\/ par heure. En considérant que c'est la seule charge variable, 
on obtient une marge sur coût variable égale à 48,20\euro.
Cette année la machine a tournée 100 heures, on obtient alors, en enlevant l'achat de grenaille 
(charge fixe):
\[
100 \times 48,20 - 600 = 4220\text{\euro}
\]
Le comptable décide ensuite de déduire de cette somme les charges fixes. 
Il commence par inclure le salaire des employés dans son calcul. 
L'employé réalisant cette 
activité étant payé 10\euro\/ de l'heure avec autant de charge sociale, le 
comptable retire $100 \times 20 = 2000 \text{\euro}$ à la somme précédente.
Le loyer de la société s'élève à 20'000\euro\/ par an, comme la salle de 
grenaillage occupe 10\% de la surface des locaux, il décide d'imputer 2000\euro\/
au grenaillage et obtient donc:
\[
4220 - 2000 - 2000 = 220\text{\euro}
\]
Son calcul semble jusque là être tout à fait sensé. 
Cependant notre comptable, qui est membre de la secte des adorateurs des coûts complets, 
souhaite distribué la totalité des charges communes aux différentes activités.
Cela inclut de manière notable le salaire de la secrétaire (40'000\euro\/ par an, 
charges comprises) et l'eau (400\euro). Comme le grenaillage représente environ 
1\% du chiffre d'affaires total, il décide de lui attribuer 1\% de ces charges, 
soit $(40\text{'}000 + 400) \times 0,\/01 = 404 \text{ page introuvable}$.
Puis il termine son calcul:
\[
220 - 404 = -184\text{\euro}
\]
Sa conclusion: on perd de l'argent !!!

Les calculs du comptables, bien que discutable sur certains points, 
sont corrects. 
Le patron a cependant une bonne raison de continuer cette activité.
\begin{itemize}
  \item "Le grenaillage nous coûte peut-être un peu d'argent 
  mais il nous amène des clients !"
\end{itemize}
Au final, cette activité a un impact positif car elle permet 
de faire plus de chiffres sur d'autres activités. 
Les clients qui viennent chez lui savent que si ils ont un 
problème avec leur pièces finies, la grenailleuse permettra 
d'enlever la peinture et d'effectuer un réusinage plutôt que 
de jetter la pièce. Certaines pièces techniques ayant une grande 
valeur, la grenailleuse est vue comme une assurance que l'on 
préfère payer, même si elle ne sert pas,
au cas où il y aurait un problème.

La morale de cette histoire est que même s'il est intéressant 
de regarder activité par activité ce que l'on gagne, il faut 
toujours savoir se remettre dans un contexte global : une activité 
peut très bien à la fois me faire perdre de l'argent mais m'en 
faire gagner beaucoup indirectement. 

Ces calculs de marges peuvent avoir une autre utilité. 
Une fois que l'on connait bien les coûts de revient des produits que 
l'on fabrique, on est en mesure d'établir des coûts standards. 
On peut ensuite ce servir de ces chiffres pour effectuer un 
management à postériori. 
Par exemple, si l'on sait que l'on utilise $n$ litres de peinture 
pour recouvrir une surface $S$ et qu'en vérifiant à la fin du mois 
l'on constate que $(k+1)$ litres de peinture ont été utilisés alors que 
seulement $kS$ mètres carrés ont été facturés c'est qu'il y a peut-être 
un problème : un pot a-t-il été malencontreusement renversé, y a-t-il une 
fuite dans la machine ?
Tout cela permet d'effectuer un contrôle et permet potentiellement de 
détecter des problèmes.

Il est également possible à partir de ce que l'on vient de faire 
d'effectuer un calcul de point mort : il s'agit d'évaluer le CA 
que l'on doit faire pour être rentable. \\
En reprenant les calculs du comptable, on a un total de 3'004 \euro\/ 
de charges fixes, et une marge sur coûts variables de 8,20 \euro\/ par heure facturée
(on inclut le salaire dans les coûts variables). En faisant le ratio, 
on obtient le nombre d'heure qu'il faut réaliser au minimum pour être rentable.
\[
3\text{'}004 \div 8,20 = 367 \text{ heures}
\]
On peut grâce à un calcul de point mort savoir quand on commencera réellement 
à gagner de l'argent. Cela permet de voir quelles sont les activités qui 
commencent à raporter le plus tôt.

Pour la petite anecdote, l'atelier a récupéré la grenailleuse 
pour une somme symbolique  
à une société située juste à côté et souhaitant s'en débarasser. 
Aujourd'hui cette dernière est un bon client de l'atelier et 
a parfois besoin d'un petit coup de grenaille !