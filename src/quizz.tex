\chapter{Quizz}


\begingroup

\renewcommand\labelitemi{\hspace*{1em}$\square$}

\begin{enumerate}
  \setlength{\itemsep}{1em}
  \item Dans le compte de résultat, les quantités sont exprimées:
    \begin{itemize}
     \item hors-taxes (HT);
     \item toutes taxes comprises (TTC).
    \end{itemize}
  \item Le compte de résultat tient compte:
    \begin{itemize}
     \item des ventes facturés;
     \item des ventes payées.
    \end{itemize}
  \item Une provision est:
    \begin{itemize}
     \item une somme mise de côté;
     \item une charge estimée future.
    \end{itemize}
  \item Quel document permet de passer du compte de résultat au cash:
    \begin{itemize}
     \item le bilan;
     \item le TFT.
    \end{itemize}
  \item Laquelle de ces quantités n'intervient pas dans le calcul du BFR:
    \begin{itemize}
     \item les stocks;
     \item les créances clients;
	 \item les dettes fournisseurs;
	 \item les salaires.
    \end{itemize}
  \item Laquelle de ces quantités dois-je aujouter au résultat d'exploitation pour retrouver l'EBE:
    \begin{itemize}
     \item les dotations aux amortissements et aux provisions;
     \item le fond de roulement.
    \end{itemize}
  \item Dans quelle partie du bilan se trouvent les machines que j'utilise pour la production:
    \begin{itemize}
     \item à l'actif;
     \item au passif.
    \end{itemize}
  \item Lorsque j'actualise une somme que je percevrai dans le futur:
    \begin{itemize}
     \item le montant actualisé est plus faible que le montant futur;
     \item le montant actualisé est plus grand que le montant futur.
    \end{itemize}
  \item Comment appelle-t-on le taux d'actualisation qui annule la VAN (valeur actuelle nette):
    \begin{itemize}
     \item le TRI;
     \item le TFT.
    \end{itemize}
  \item Lorsque je fais mes calculs de marges, en plus des charges fixes spécifiques et non spécifiques, je dois tenir compte  :
    \begin{itemize}
     \item des charges variables;
     \item des charges déductibles d'impôts.
    \end{itemize}
\end{enumerate}

\textbf{Bonus:} Laquelle de ces citations n'est pas de Pascal:
\begin{enumerate}
  \item D'après le pêcheur Pornicais, ``Une voiture avance dans le brouillard''.
  \item ``La prise de température n'a jamais guéri qui que ce soit.''
  \item ``Il vaut mieux être précisement faux, mais globalement juste.''
  \item ``Qui a le template ?''
\end{enumerate}

\endgroup