
\chapter{Cash et Résultat \\ tu ne confondra point}

\chapterprecishere{``Tout le monde peut faire une erreur [...]; mais cette 
erreur ne devient une faute que si vous refusez de la corriger''\par\raggedleft--- Mitth'raw'nuruodo}

Il est 8h, je viens d'arriver au boulot.
Les vacances approchent, la plupart des gens sont sur les nerfs. 
Je me dirige vers la salle de repos pour me préparer un thé, 
la machine à café étant en panne depuis presque deux semaines, 
nous avons recours à une bouilloire pour faire chauffer l'eau.
En m'approchant, j'entends que le volume dans la salle n'est 
certainement pas reposant. C'est Bernard, le commercial, 
qui rage sur la machine qui n'a toujours pas été réparée. 
Les personnes présentent ne savent plus où se mettre... 
Prenant mon courage à deux mains, je décide d'essayer de 
l'écouter pour qu'il se calme.
\begin{itemize}
  \item "Un problème Bernard ? "
  \item "Bah oui ça fait chier la machine qui est toujours en 
  panne ça fait trois semaines !"
\end{itemize}
Étant diplomate, je décide de ne pas lui faire remarquer que 
la machine n'est en panne que depuis seulement deux semaines, et pas 
trois.
\begin{itemize}
  \item "Oui, je sais. Mais t'inquiètes elle va bientôt être 
  réparée, c'est sûre."
  \item "Et bah non justement, j'ai été voir le patron et il 
  a dit qu'on avait pas l'argent pour réparer la machine, t'imagines !"
\end{itemize}
C'est à se moment là que j'ai décidé d'appliquer la méthode DG, 
qui consiste à écouter la personne râler jusqu'à la dernière 
goutte. On peut alors discuter plus calmement. 
Il continua en affirmant que c'était scandaleux, qu'on avait 
demandé aux commerciaux d'augmenter les ventes d'au 
moins 5\%, et que c'est ce qu'ils avaient fait et même 
mieux.
\begin{itemize}
  \item "7\% on a fait, 7\% tu te rends compte !"
\end{itemize}
Il me demanda si j'avais assisté à la présentation des 
résultats de l'entreprise hier, ce à quoi je répondis oui; et 
continua en m'expliquant que le chiffre d'affaires et le 
résultat avaient augmentés comme prévu. 
Il ne comprenait donc pas pourquoi on n'avait pas l'argent 
pour réparer cette "foutue" machine à café. 
Il estimait que lui et les autres commerciaux méritaient 
d'être mieux traités que cela après les efforts qu'ils avaient 
fait.

Il faut dire que la présentation de la veille ne présentait 
que les aspects positifs des résultats de la société, et 
ceux en partit pour ne pas démoraliser les troupes qui avaient 
fait des efforts mais également peut-être par lâcheté et 
habitude du manager, un ancien d'Audencia, qui a l'habitude de 
ne montrer que ce qui est plaisant. 
Le résultat de la boîte s'est en effet amélioré mais pas sa 
trésorerie, ce qui explique qu'il n'y ai plus d'argent pour 
le café... 

Pour comprendre ce qu'il se passe, il faut étudier trois 
documents comptables: le compte de résultat, le bilan, et le 
tableau des flux de trésorerie. 
Les normes françaises imposent aux entreprises de soumettre 
les deux premiers à l'Etat.

Le compte de résultat est un document dont le but est de 
refléter l'activité de la société sur une année (un \emph{exercice});
le bilan est là quant à lui pour 
présenter ce que possède la société, en terme d'argent mais 
également de biens matériels, et comment elle l'utilise.

Un exemple de ces documents est présenté un peu plus en détails en annexe.

Le compte de résultat est découpé en trois grandes parties : 
la partie relative au résultat d'exploitation, celle relative 
au résultat financier et enfin une dernière partie sur le 
résultat exceptionnel. La somme de ces trois résultats forme 
le résultat de l'exercice.

Le bilan est découpé en deux parties : l'actif (ce que j'ai) 
et le passif (mes ressources, i.e. ce que je dois).
Autrement dit, le bilan présente les ressources de l'entreprise 
et comment elle les utilise; le bon sens Pornicais impose donc que 
le passif soit égal à l'actif\footnote{On parle aussi de bon sens paysan (BSP)}.

L'une des notions les plus importantes est que le compte 
de résultat ne représente pas du cash. En effet, sont 
enregistrés dans le compte de résultat toutes les factures dès 
leur émission et tous les achats dès leur réception : le 
compte de résultat ne prend pas en compte le paiement effectif. \\
\hspace*{\parindent}Une autre raison est qu'il est inscrit dans le compte de 
résultat les dotations aux amortissements : dès que j'achète un bien 
d'une valeur supérieure à 500\euro\/, je ne le fais pas passer 
entièrement dans le compte de résultat mais je l'étale sur 
plusieurs exercices (cela permet de prendre en compte le fait 
que j'utilise ce bien sur plusieurs exercices et qu'il perd de sa 
valeur).

Le tableau des flux de trésorerie permet de passer du 
résultat au cash.

\begin{table}[h]
\renewcommand{\arraystretch}{1.2}
\footnotesize
\centering
\begin{tabular}{|l|c|}
\hline                                                                                                                                                                
  CAF    & 120 \\
\hline                                                                                                                                                              
  Variation du BFR  & (20) \\
\hline                                                                                                                                                              
  Cash Flow d'Exploitation & 100 \\
\hline  
  Flux de trésorerie d'investissement & 10 \\
\hline     
Cash Flow Libre & 110 \\                                                                                                                                                                                                                                                                                              
\hline                                                                                                                                                                
Flux de trésorerie de financement & (120) \\
\hline     
Variation de trésorerie & (10) \\                                                                                                                                                                                                                                                                                              
\hline  
Trésorerie au début de l'exercice & 5 \\                                                                                                                                                                                                                                                                                              
\hline  
Trésorerie à la fin de l'exercice & (5) \\                                                                                                                                                                                                                                                                                              
\hline  
\end{tabular}
\label{tftex1}
\caption{Tableau des flux de trésorerie simplifié}
\end{table}

Les délais de paiement des clients et des fournisseurs 
introduisent la notion de besoin en fonds de roulement (BFR) 
et de fonds de roulement. Concrètement, le BFR est la 
trésorerie qu'il me faut à l'avance pour pouvoir payer mes 
fournisseurs avant de recevoir le paiement de mes clients.
Son expression est la suivante:
\begin{equation*} \label{eq:BFR}
\mathrm{BFR} = \text{Stocks} + \text{Créances clients} - \text{Dettes fourniseurs}
\end{equation*}
Le fond de roulement est le cash disponible pour combler le 
BFR. Il doit donc être supérieur au BFR pour éviter de devoir 
s'endetter.
\[
\mathrm{FR} = \text{Capitaux permanents} - \text{Actifs immobilisés}
\]
où les capitaux permanents sont la somme des capitaux propres et des dettes à moyen et long terme.

Ces deux quantités sont calculables à partir du bilan.

Comme on peut le voir pour notre société, c'est la variation 
du besoin en fonds de roulement qui fait passer la trésorerie 
dans le rouge. Les ventes ont certes augmentées, mais comme 
le besoin de fond de roulement a lui aussi augmenté la 
situation financière s'est dégradée. 
L'augmentation du BFR peut avoir pour origine: 
\begin{itemize}
 \item une augmentation des stocks;
 \item une augmentation des créances clients;
 \item une diminution de la dette fournisseurs (e.g. on paye plus vite qu'auparavant);
 \item une combinaison des trois.
\end{itemize}

Cette exemple montre deux choses importantes:
\begin{itemize}
 \item il est important de maîtriser sont BFR (composantes par composantes);
 \item le résultat ne se traduit pas immédiatement en cash à cause des délais de paiement.
\end{itemize}

L'évaluation de la santé d'une société peut se faire de 
plusieurs manières. En comptabilité analytique, on utilise 
les soldes intermédiaires de gestion (c.f.\/ table \ref{table:sig} annexe \ref{chapter:sig}). 
En analyse financière, on va utiliser d'autres indicateurs que l'on peut 
construire à partir des documents comptables. 

Il est possible de faire une analyse très rapide en utilisant seulement trois indicateurs 
que l'on trouve au début, au milieu et à la fin du compte de résultat. Ces indicateurs 
comptables et leurs équivalents financiers sont présentés dans la table \ref{table:3indicateurs}.

\begin{table}[h]
\renewcommand{\arraystretch}{1.2}
\footnotesize
\centering
\begin{tabular}{|c|c|c|}
\hline
  Comptable & Financier & Sens \\                                                                      
\hline                                                                                                      
  CA & VA & Puissance \\                                                                    
\hline                                                                                                      
  REX & EBE & Moteur \\                                                                              
\hline                                                                                                          
  RNS & CAF & Développement \\                                                                                         
\hline                                                                                                      
\end{tabular}                                                                                                                                                                                                       
\caption{Analyse en 3 indicateurs}       
\label{table:3indicateurs}                                                    
\end{table}

Pour finir ce chapitre et à nouveau appuyer sur le fait que le \textsc{cash} est différent 
du résultat, nous ajouterons simplement que chaque année, une part non négligeable des entreprises 
qui sont dissoutent ont pourtant un résultat positif (mais sont à cours de \textsc{cash})...
