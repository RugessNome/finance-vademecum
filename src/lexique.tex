\chapter{Lexique}

\newcounter{defcounter}
\newcommand{\definition}[2]{\refstepcounter{defcounter}\label{def:#1}\textbf{#1 :} #2\vspace{0.25em}}
\newcommand{\ldefinition}[3]{\refstepcounter{defcounter}\label{def:#1}\textbf{#2 :} #3\vspace{0.25em}}
\newcommand{\refdef}[1]{\hyperref[def:#1]{#1}}

\setlength{\parskip}{0em}
\setlength\abovedisplayskip{0.5em}
\setlength\belowdisplayskip{-0.5em}

\definition{Amortissement}{constatation comptable de la perte de valeur subie par un bien du fait 
de son utilisation ou de sa détention par l’entreprise.
}


\definition{Besoin en fonds de roulement}{ou BFR;
ressources financières que l'entreprise doit avancer pour couvrir les décalages de paiement entre 
les clients et les fournisseurs ainsi que la création d'un stock; voir explication page \pageref{eq:BFR}.
}

\definition{BFR}{
voir \refdef{Besoin en fonds de roulement}.
}

\definition{Bilan}{le bilan comptable est un document qui synthétise ce que possède une entreprise à 
un moment donné (l'actif) et ses ressources (le passif).
}

\definition{CA}{
voir \refdef{Chiffre d'affaires}.
}

\definition{CAF}{
voir \refdef{Capacité d'autofinancement}.
}

\definition{Capacité d'autofinancement}{ou CAF;
Capacité d'une entreprise à investir et rembourser ses emprunts; 
est en général un bon indicateur de la capacité maximale que l'entreprise peut 
emprunter (de l'ordre de quelques fois la CAF).
\begin{equation*}
\text{CAF} = \text{RNS} + \text{Dotations (amortissements, provisions)}
\end{equation*}
}

\definition{CAPEX}{\nfw{capital expenditure};
dépenses d'investissement, i.e.\/ immobilisations.
}

\definition{Capitaux propres}{CP;
}

\definition{Cash Flow libre}{CFL;
}

\definition{Chiffre d'affaires}{
Ventes hors taxes de l’année sans tenir compte des encaissements.
}

\definition{Compte de résultat}{
Document comptable rendant compte sur une période, 
appelée exercice comptable, des charges et produits hors taxes de la société.
}

\definition{Days Purchasing Outstanding}{DPO;
Dettes fournisseurs en nombre de jours de CA.
}

\definition{Days Sales Outstanding}{DSO;
Créances clients en nombre de jours de CA.
}

\definition{DPO}{
voir «\refdef{Days Purchasing Outstanding}»
}

\definition{DSO}{
voir «\refdef{Days Sales Outstanding}»
}

\definition{EBE}{voir «\refdef{Excédent brut d'exploitation}»
}

\definition{EBITDA}{\nfw{Earnings Before Interest Tax Depreciation}; voir «\refdef{Excédent brut d'exploitation}»
}

\definition{Excédent brut d'exploitation}{EBE;
bénéfice dégagé par l'activité principale de l'entreprise; en anglais EBITDA
\begin{equation*}
\text{EBE} = \text{REX} + \text{Dotations (amortissements, provisions)}
\end{equation*}
}

\definition{Excédent de trésorerie d'exploitation}{ETE;
flux de trésorerie généré entre les produits effectivement encaissés et les charges d'exploitation 
décaissés.
\begin{equation*}
\text{ETE} = \text{CAF} + \text{variation BFR}
\end{equation*}
}

\definition{Fonds de roulement}{FR;
argent que l'on est capable d'avancer pour combler le 
\refdef{Besoin en fonds de roulement}.
}

\definition{Immobilisation}{bien (matériel ou non) non destiné à la vente dont le coût 
est amorti suivant sa durée de vie.
}

\definition{Marge brute}{différence hors taxes entre le prix de vente et le prix de revient.
}

\definition{OPEX}{\nfw{operational expenditure};
dépenses d'exploitation.
}

\ldefinition{pl}{P \& L}{\nfw{profit \& loss}; voir «\refdef{Compte de résultat}».
}

\definition{Provision}{charge probable que l'on rattache à un exercice comptable 
pour mieux refléter l'activité de la société; son montant est estimé.
}

\definition{Réserves}{anciens bénéfices dégagés par la société qui n'ont pas été redistribués; 
se trouve au passif dans le bilan.
}

\definition{Résultat d'exploitation}{REX;
Résultat généré par l'activité principale de la société.
}

\definition{ROI}{\nfw{return on investment}.
}

\definition{Taux d'endettement}{ratio entre la dette bancaire et les capitaux propres.
}

\definition{TRI}{taux de rentabilité interne; valeur du taux d'actualisation 
qui annule la VAN.
}

\definition{Valeur ajoutée}{VA;
\begin{equation*}
\text{VA} = \text{CA} - \text{Achats}
\end{equation*}
}

\definition{VAN}{valeur actualisée nette; cumul des flux de trésorerie actualisés.
}


\setcounter{tocdepth}{1}
