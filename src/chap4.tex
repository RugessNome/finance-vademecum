
\chapter{Un choix difficile ?}

\chapterprecishere{``Pour comparer deux quantités, on regarde le signe 
de la différence''\par\raggedleft--- François Ranty}

Les évènements relatés au chapitre précédent montrent à 
la fois l'importance des calculs de marges (évaluation 
de la rentabilité / des performances, coûts standards, etc...\/)
mais également la difficulté d'intégrer les charges fixes non 
spécifiques dans ces calculs... \\
\hspace*{\parindent}Cela peut être le sujet de longues discussions et peut parfois 
amener à des résultats contre-intuitifs.
\begin{itemize}
 \item "Tiens, par exemple", me dit le comptable.
\end{itemize} 
"Si je refais les calculs en imputant les charges fixes non 
spécifiques de manière équiproportionnelles à toutes les activités, 
puis que je supprime les activités ayant une marge totale 
négative; les marges de toutes mes autres 
activités diminuent fortement, et deviennent parfois elles aussi négatives..."
\begin{itemize}
 \item "C'est les coûts fixes mon vieux", lui répondis-je.
\end{itemize} 
En fait, quand il s'agit de programmer l'arrêt d'une activité, il est 
à priori préférable de choisir celle qui a la marge sur coûts variables 
la plus négative: en effet, plus on effectuera de vente sur cette activité, 
plus cela dégradera le résultat. \\
\hspace*{\parindent}L'un des autres problèmes, lorsque l'on raisonne en terme de projets et 
non plus en terme d'activités, est que l'on a tendance à attribuer certaines 
charges plus facilement à certains projets battant de l'aile pour 
en favoriser d'autres. 
C'est un peu le principe de la méthode JGSC\footnote{J'en garde sous le coude.}.
Si cela permet de faire plaisir au management, cela peut cependant avoir 
des conséquences assez néfaste, notamment sur le calcul des coûts standards...

Ces discussions sur les marges sont très intéressantes, mais un 
autre problème nécessite notre attention.
Il y a quelques temps, une machine sur la chaîne de 
production s'est déréglée et la personne en charge de 
vérifier le bon fonctionnement s'en est rendu compte 
un peu tard. \\
\hspace*{\parindent}Résultat : 1000 pièces non-conformes ont été usinées. \\
\hspace*{\parindent}Ce sont des pièces élaborées destinées à des industries où 
les normes sont assez strictes. 
Elles sont vendues à 60 \euro\/ 
(+ 5 \euro\/ d'expédition) pour un prix de revient de 45 \euro.
Les retours clients coûtent 10 \euro\/ pièce mais sont très rares 
sur cette référence. \\
\hspace*{\parindent}Pour tenter de rattraper le coup, nous avons déjà effectué 
un réusinage sur les pièces (10 \euro\/ de retouche par pièce); 
mais notre client n'est pas satisfait et il ne nous reste que 
la solution de sous-traitance pour résoudre le défaut 
(coût : 25\euro\/ par pièce). \\
\hspace*{\parindent}Ajouter à cela le fait que le client nous facturera de toute manière 
une pénalité de retard d'un montant de 20 \euro\/ par pièce, 
nous sommes tentés de choisir
une autre solution qui se présente à nous: on peut mettre la pièce 
au rebut et récupérer pour 20 \euro\/ de matière, et 
annuler la commande. \\
\hspace*{\parindent}Roger, un fidèle ouvrier, commence alors à faire les calculs 
dans les deux situations pour savoir ce qu'il faut faire.
\begin{itemize}
 \item "On sous-traite !"
 \item "Quoi !?", me dit Roger.
\end{itemize} 
Ce dernier a du mal à masquer son étonnement et me demande 
aussitôt comment j'ai fait tous les calculs si vite et de tête.
Je lui explique que pour résoudre ce genre de problème il n'est 
pas nécessaire de prendre en compte tous les éléments. 

On peut prendre une décision très rapidement en appliquant 
la méthode FDP\footnote{A ne pas confondre avec fils 
de la plage.} : cette méthode consiste à ne prendre en 
compte que les éléments qui sont \emph{futurs}, \emph{différentiels} 
et \emph{pertinents}. 
Voyons ce que cela donne pour notre cas dans le tableau suivante.

\begingroup
\scriptsize
\begin{center}
\begin{tabular}{llcccc}
  \'Element        & F        & D        & P         & Opt.\/ 1 & Opt.\/ 2 \\
  \hline
  Production pièce &          &          & $\times$  &         &         \\
  Vente pièce      & $\times$ & $\times$ & $\times$  &  65     &         \\
  Retouche         &          &          &           &         &         \\
  Sous-traitance   & $\times$ & $\times$ & $\times$  & (25)    &         \\
  Pénalité         & $\times$ &          & $\times$  &         &         \\
  Rebut            & $\times$ & $\times$ & $\times$  &         & 20      \\
  Retours-client   & $\times$ & $\times$ &           &         &         \\
  \hline
  \multicolumn{4}{l}{\hspace{1em} Total}             & 40      & 20\\
\end{tabular}
\end{center}
\endgroup

Comme on peut le voir, l'option numéro 1 (la sous-traitance) obtient 
le meilleur score et c'est donc elle qu'il faut choisir. 
Les calculs sont beaucoup plus simples que si on avait tout 
pris en compte. 
Pour savoir quelle option est la meilleure, il n'est pas nécessaire de 
prendre en compte les éléments qui nous ont déjà coûté de l'argent, 
ni ceux qui sont communs aux deux options. 
Enfin, les éléments peu pertinents (ici les retours-client qui sont 
très rares) n'ont pas non plus besoin d'être considérés. 


Il est cependant important de comprendre que si cette méthode 
permet de prendre rapidement une décision, elle n'empêche pas 
de faire ensuite un calcul sur l'impact dans le résultat. 
En effet, même si ici la première option est la meilleure, 
une fois tous les coûts pris en comptes (la production des 
pièces, la pénalité de retard, etc...\/), on obtient bien 
une perte.
