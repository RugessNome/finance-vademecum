
\chapter{Un investissement rentable ?}


\chapterprecishere{``Un tiens vaut mieux que deux tu l'auras.''\par\raggedleft--- \textup{Le Petit Poisson et le Pêcheur}, La Fontaine}

Gérard, fidèle employé de notre petite société de
fabrication est fatigué de travailler sur une machine qui n'est 
plus à la pointe de la technologie. 
Voulant améliorer la situation, il décide de faire des 
recherches et finit par trouver une machine qui permettrait de 
produire autant et pour moi cher.
En plus de rendre la vie des travailleurs plus agréable, 
cette machine pourrait donc également permettre de faire 
des économies. \\
\hspace*{\parindent}Comme Gérard sait bien que le patron est souvent occupé, 
il décide de se construire un petit dossier pour appuyer son idée.
Il fait ses calculs et en déduit que l'investissement serait 
rentable en moins de 5 ans. \\
\hspace*{\parindent}Fier comme un bar-tabac, 
il court donc voir son patron pour lui 
soumettre l'idée. Il lui montre tout son petit dossier: la 
fiche technique de la machine, le coût de l'investissement 
et les économies qu'il va pouvoir réaliser chaque année.
Le patron, qui n'est pas né de la dernière pluie, sait que 
tout n'est pas si simple. Il suspecte que le calcul de 
son employé ne soit pas tout à fait juste. 
Cependant, comme il ne veut pas se mettre ses employés à dos, 
il décide de renvoyer le poisson vers le monsieur finance 
de l'entreprise. Après tout, ce dernier n'est déjà pas apprécié 
par la majorité des employés, autant que ce soit lui qui 
annonce les mauvaises nouvelles.
\begin{itemize}
  \item "Oui, ça pourrait être intéressant, va donner ça à 
  Francis qu'il nous dise si c'est faisable"
  \item "Ok chef !"
\end{itemize}
\hspace*{\parindent}L'employé va alors voir le comptable, 
frappe à la porte de son bureau et fini par être accueilli par le 
comptable qui visiblement est dérangé (il était probablement 
en train d'effectuer son activité favorite, enregistrer des 
pièces comptables).
Gérard, voulant alors s'attribuer tout le mérite, s'adresse 
au comptable comme suit \\
\begin{itemize}
  \item "J'ai réfléchi à un investissement qui pourrait nous faire 
    économiser pas mal d'argent et le chef m'a dit de venir vous apportez ça,
	voilà en faite il s'agirait..."
  \item "Posez ça sur mon bureau je regarderais plus tard", 
    dit le comptable pour couper court à Gérard, voyant que 
    ce dernier allait probablement s'étendre en explications.
\end{itemize}
\hspace*{\parindent}Ce dernier, ne voulant pas risquer d'énerver le comptable, 
déposa son petit dossier sur le bureau et partit en se disant 
que tout allait aller comme sur des roulettes.
\hspace*{\parindent}Quelques jours plus tard, n'ayant pas de nouvelles,
Gérard va voir le comptable pour lui en demander. 
Il ressort de son bureau quelques minutes plus tard, bien 
énervé: le comptable a rejeté son idée ! 


Pour comprendre comment deux personnes en sont arrivés 
à des conclusions différentes, il faut étudier la façon 
dont ils ont chacun fait leurs calculs.

Gérard, lui, à décidé de voir les choses simplement: le 
coût d'achat et de mise en service de la nouvelle machine 
serait de 100 k\euro\/; cette dernière permettrait sur 5 ans 
d'économiser 24 k\euro\/ par an. Au final, 
\[ 5 \times 24 - 100 = 20 > 0 \]
donc d'après Gérard le projet est rentable.

Le financier, quant à lui, sait que ce n'est 
cependant pas aussi simple que cela ! 
Lorsqu'il a fait son calcul, ce dernier à pris en compte 
le fait que $1$ euro dans 5 ans ce n'est pas $1$ euro aujourd'hui. 
Pour montrer cela, nous allons prendre l'exemple simple d'un placement à 
la banque. Supposons que je place en banque $x$ euros. 
Chaque année, la banque va me reverser $\alpha$ \% d'intérêts.
Ainsi, à l'issue de la première année, je n'aurais plus $x$ 
euros en banque mais $x \times (1+\alpha)$, quantité 
que nous appellerons $F_1$. 
\hspace*{\parindent}Ainsi, après seulement une année à la banque, 
j'ai déjà plus d'argent qu'au départ !
Si je laisse cet argent à la banque, et que les taux restent 
constant, je gagnerais chaque année une somme vérifiant:
\[ F_n = x \times (1+\alpha)^n \]
Ceci montre bien qu'$1$ euro dans 5 ans, ce n'est pas $1$ euro 
aujourd'hui.
On peut évidemment faire autre chose de cet argent 
(ce n'était qu'un exemple), mais dans 
tous les cas, on peut espérer qu'il nous rapporte. 
Ceci implique qu'il est nécessaire "d'actualiser" les 
flux de trésorerie futurs pour faire un calcul plus juste.

La formule précédente nous permet d'estimer la valeur future 
de notre argent. En l'inversant, on peut calculer l'équivalent 
aujourd'hui d'un montant que l'on recevrait dans le futur.
\[ F_0 = \frac{F_n}{(1+\alpha)^n} \]
Dans cette formule, $\alpha$ est appelé le \emph{taux 
d'actualisation}, il permet d'actualiser la valeur d'une somme future.
On est alors en mesure de ré-estimer le coût de notre 
investissement.

La taux d'actualisation est en général pris égal au \emph{WACC} (\textit{weighted average cost of capital}, ou 
CMPC en français, coût moyen pondéré du capital).
Ce dernier se calcul de la manière suivante:
\begin{equation*}
\label{eq:WACC}
\mathrm{WACC} = \frac{C_p R + D_f I (1- t_i) }{C_p + D_f}
\end{equation*}
Avec:
\begin{itemize}
  \item $C_p$, les capitaux propres;
  \item $R$, le coût des capitaux propres, i.e.\ la rentabilité attendue par les actionnaires;
  \item $D_f$, les dettes financières;
  \item $I$, les intérêts des dettes financières;
  \item $t_i$, le taux d'imposition.
\end{itemize}

Moralement, ce quotient correspond au taux d'actualisation 
qui permet de satisfaire les investisseurs et de payer les 
intérêts des dettes; ces derniers sont coefficientés 
par le taux d'imposition car les intérêts font diminuer 
le résultat et donc l'impôt payé (ils sont en quelques sortes 
déductibles d'impôts).

Par exemple, pour $\alpha = 5\%$, on obtient.

\begin{table}[h]
\small
\centering
\begin{tabular}{|r|c|c|c|c|c|c|}
\hline
 Année & 0 & 1 & 2 & 3 & 4 & 5 \\
 Flux  & -100 & 24 & 24 & 24 & 24 & 24 \\
 Actualisé & -100 & 22,86 & 21,77 & 20,73 & 19,74 & 18,80 \\
 Cumul & -100 & -77,14 & -55,37 & -34,64 & -14,90 & 3,91 \\
\hline
\end{tabular}
\caption{Flux de trésorerie actualisé}
\end{table}

On se rend compte que ce calcul est tout de suite moins 
avantageux pour ce qui est de la rentabilité du projet: 
les ROI est ici un peu en dessous des 4\%, ce qui est 
assez faible.

La formule utilisé pour actualiser nos flux de trésorerie 
reste cependant un peu simpliste: elle suppose par exemple 
que le taux d'actualisation reste constant au cours du temps 
ce qui n'est pas réellement le cas. 
C'est pourquoi l'on effectue en général les calculs avec 
différents taux d'actualisation, pour voir à quel point 
l'investissement est intéressant. 
On s'intéresse en 
particulier à la valeur du taux d'actualisation, le TRI, qui 
annule les bénéfices. Plus le TRI est élevé, plus 
l'investissement est rentable.

Attention cependant, il est parfois préférable d'avoir 
un TRI relativement faible mais sur un projet d'une durée 
courte, plutôt qu'un TRI élevé mais sur un temps beaucoup 
plus long.

Ici, le calcul donne un TRI d'environ 6,4\%. 
Le projet a probablement été rejeté car cette valeur est 
inférieure au WACC de la société.


